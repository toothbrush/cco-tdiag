\documentclass[a4paper]{article}

\title{Design documentation \texttt{tdiagrams}}
\date{\today}
\author{Paul van der Walt}

\begin{document}

\maketitle

\section{Introduction}

This document explains briefly what the layout of the source code is of the
tools which constitute the \texttt{tdiagrams} suite. Here the changed or added
files are also presented, since this work is based on the tdiagrams-0.0.4
package by Stefan Holdermans. 

\section{General info}

The distribution tdiagrams-0.0.4 already included implementations of the parser and pretty printer (\texttt{parse-tdiag} and \texttt{pp-picture}, respectively), so all that remained to be done was to implement the tools \texttt{tc-tdiag}, which should type check a T-Diagram, and \texttt{tdiag2picture} which would generate \LaTeX\ code for a given T-Diagram.

The choice was made to implement these functionalities in two separate Attribute Grammar (AG) files, in the already-present \texttt{Diag}-section. This would enable re-use of the already existing data structures for Diag, among other things. 


\section{Files}




\end{document}
